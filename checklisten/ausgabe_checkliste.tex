\documentclass[a4paper]{scrartcl}
\usepackage[utf8]{inputenc}
\usepackage[T1]{fontenc}


\begin{document}
\section*{Bei Ausgabe beachten}
\label{sec:bei-ausgabe-beachten}
\begin{itemize}
\item Vor der Ausgabe ins Kochbuch sehen, sich die Rezepte des Tages ansehen.
\item Nach der Zeltföffnung den Tisch abwischen.
\item Die Kochgruppe begrüßen, die Kochgruppenkarte entgegennehmen.
\item Die Helfer losschicken, die Kisten zu holen.
\item Kühlkiste
  \begin{itemize}
  \item Für die Kühlkiste Rollwagen verwenden.
  \item Kochgruppenkarte mitgegeben.
  \item Kühlkistenkarte bei Ausgabe einsammeln 
  \end{itemize}
\item Die Seite der Kochgruppe am Rechner aufsuchen, die Gruppe über ihren Kistenstand informieren.
\item Spezielle Infos weitergeben:
  \begin{itemize}
  \item Wurde etwas ersetzt?
  \item Gibt es Hajk-Verpflegung?
  \item Muss etwas anderes beachtet werden (Einweichen?)
  \end{itemize}
\item Bei Kistenübergabe: \begin{itemize}
  \item Kochgruppenkarte zurückgeben
  \item Fragen, ob die Gruppe etwas dalassen möchte (Was über den Tresen ging, darf nicht zurückgenommen werden!)
  \item Kistenstand einbuchen
  \end{itemize}
\item Kühlwarenrückläufer direkt in die Kühlung bringen lassen.
\item Zu wenig Essen? $\Rightarrow$ Schichtleiter rufen.
\item Möchten gerne noch etwas? Rückläufer können direkt mitgegeben werden.  Für alles weitere den Tresen empfehlen.
\end{itemize}

\newpage
\section*{Nach der Ausgabe}
\begin{itemize}
\item Zelt schließen
\item Tische abwischen
\item neue Kisten verteilen
\item Müll wegbringen lassen
\item Unter die Tische schauen, besonders bei den Waagen, Fegen lassen.
\item Rückläufer sortenrein sortieren lassen
\item Mit Schichtleiter und den anderen Packstraßenleitern die Waren ans Lager übergeben.
\end{itemize}




\end{document}



%%% Local Variables:
%%% mode: latex
%%% TeX-master: t
%%% End:
