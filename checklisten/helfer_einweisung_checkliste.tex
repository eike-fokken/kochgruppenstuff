\documentclass[a4paper]{scrartcl}
\usepackage[utf8]{inputenc}
\usepackage[T1]{fontenc}
\usepackage[ngerman]{babel} 

\begin{document}
\section*{Helfer-Einweisung}
\begin{itemize}
\item Blickkontrolle: Geschlossenes Schuhwerk, saubere Kleidung
\item
  \begin{itemize}
  \item War jemand innerhalb der letzten 3 Tage krank? Falls ja: Muss heim
  \item Fragt Symptome ab: Durchfall, Schnupfen, Halsschmerzen, Husten, Kratzen im Hals
  \item Ist jemand allergisch gegen Desinfektionsmittel: Dann stattdessen Handschuhe tragen
  \item Hat jemand offene Wunden: Abkleben, bei Handwunden: Handschuhe
  \end{itemize}
\item Erklärung: Hände waschen und desinfizieren, bzw.~Handschuhe (bei Allergie gegen Desi).
\item Hände gründlich einseifen (Zwischenräume, Kuppen, Handflächen, Daumen, Handrücken)
\item Abwaschen
\item Regeln sind
  \begin{itemize}
  \item Maske tragen
  \item Nicht in der Packstraße essen
  \item Lebensmittel und Kisten nicht auf den Boden (Mindestens Palette).
  \item Nach dem Wiegen Werkzeug und Umgebung (Waage, Schaufeln, Tische) abwischen (Allergien...)
  \item Was runterfällt, ist Müll.
  \item Nach jedem Produkt Hände desinfizieren (Allergene).
  \end{itemize}
\item Bei Fragen immer an den Packstraßenleiter wenden.
\item In der Packstraße bleiben.
\item Das Lager nicht betreten.
\end{itemize}



\end{document}



%%% Local Variables:
%%% mode: latex
%%% TeX-master: t
%%% End:
